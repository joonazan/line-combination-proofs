\documentclass[english, 12pt, a4paper, sci, utf8, a-1b, online]{aaltothesis}

\usepackage[numbers]{natbib}
\usepackage{graphicx}
\usepackage[most]{tcolorbox}
\usepackage{amsfonts,amssymb,amsbsy,amsmath}

\usepackage{minted}
\usemintedstyle{friendly}
% horrible hack to hide pygment's complaints about unicode
\AtBeginEnvironment{minted}{\dontdofcolorbox}
\def\dontdofcolorbox{\renewcommand\fcolorbox[4][]{##4}}

\degreeprogram{Computer, Communication and Information Sciences}
\major{Computer Science}
\code{SCI3042}

\univdegree{MSc}
\thesisauthor{Joonatan Saarhelo}
\thesistitle{Fast and Correct Round Elimination}
\place{Espoo}
\date{2021}

\supervisor{Prof.\ Jukka Suomela}
%\advisor{}

%% \uselogo{aaltoRed|aaltoBlue|aaltoYellow|aaltoGray|aaltoGrayScale}{?|!|''}
\uselogo{aaltoRed}{''}

\keywords{For keywords choose\spc{}concepts that are\spc{}central to your\spc{}thesis}

\thesisabstract{
Your abstract in English. Cannot contain special characters, linebreak or paragraph
break characters as it is written into the metadata.
}

%% Copyright text. Copyright of a work is with the creator/author of the work
%% regardless of whether the copyright mark is explicitly in the work or not.
%% You may, if you wish, publish your work under a Creative Commons license (see
%% creaticecommons.org), in which case the license text must be visible in the
%% work. Write here the copyright text you want. It is written into the metadata
%% of the pdf file as well.

\copyrighttext{Copyright \noexpand\copyright\ \number\year\ \ThesisAuthor}
{Copyright \textcopyright{} \number\year{} \ThesisAuthor}

\begin{document}

\makecoverpage{}

\makecopyrightpage{}

\begin{abstractpage}[english]
  \abstracttext{}
\end{abstractpage}

\newpage

\thesistitle{Nopea ja virheetön kierroseliminaatio}
\keywords{Vastus, resistanssi, lämpötila}

\begin{abstractpage}[finnish]
  Tiivistelmässä on lyhyt selvitys
  kirjoituksen tärkeimmästä sisällöstä: mitä ja miten on tutkittu,
  sekä mitä tuloksia on saatu. 
\end{abstractpage}

\thesistableofcontents{}


\mysection{Symbols and abbreviations}

\subsection*{Symbols}

\begin{tabular}{ll}
\end{tabular}

\subsection*{Abbreviations}

\begin{tabular}{ll}
RE         & round elimination
\end{tabular}

\cleardoublepage{}
\section{Introduction}

%% Leave page number of the first page empty
\thispagestyle{empty}

\clearpage
\section{Round Elimination}

Brandt et al. introduced round elimination in 2019\cite{speedup}. Dennis Olivetti wrote an implementation of it called Round Eliminator\cite{RE}. Round Elimination has been used to prove bounds on time complexity for various problems in the LOCAL model\cite{tc1, tc2, tc3}.

\subsection{Locally Checkable Labeling}

Round Elimination operates on an encoding of Locally Checkable Labeling (LCL) problems. LCL-problems are edge coloring problems on biregular trees. One of the bipartitions is called the active side and the other is the passive side. Each side has a set of multisets of colors. Each multiset describes one acceptable way to color the edges of that type of vertex. A solution to an LCL-problem is a coloring where every vertex's edges are colored in an acceptable way.

In Round Eliminator, problems are represented as \emph{lines}. Lines are a kind of shorthand notation that compresses multiple multisets into one line. Each line is a multiset of sets and represents or \emph{generates} all the multisets obtained by choosing one color from each set.\cite{RE}

\begin{figure}[h]
\centering
\begin{tcolorbox}[width=.2\textwidth, nobeforeafter, title=active side]
A A A A \\
B B B B \\
C C C C
\end{tcolorbox}
\begin{tcolorbox}[width=.2\textwidth, nobeforeafter, title=passive side]
A BC \\
B C
\end{tcolorbox}
\caption{Three-coloring in Round Eliminator's shorthand}
\end{figure}

\subsection{Round Elimination}

As the name suggests, Round Elimination takes a problem $P$ that can be solved in $r$ rounds and outputs a new problem $P'$ that is solvable in $r-1$ rounds. But what is $P'$ relationship to $P$?

While $P$ assigns a color to each edge, $P'$ assigns a set of colors each edge. That set contains the colors that the edge could have. There is uncertainty because in $r-1$ rounds some information that affects the solution of $P$ is out of reach.

The new active side consists of all lines that only generate configurations present in the old passive side, excluding lines that are subsets of other lines. The new passive side consists of all configurations made of the new active sides' symbols that contain a configuration from the old active side.\cite{DA2020}

Written as a computer program, the above definition is a very concise and slow implementation of round elimination. To prove the correctness of a round elimination algorithm it suffices to prove that its output contains the same configurations as the reference implementation's output.

\section{Minimization}

Round Elimination produces a gigantic number of configurations. Especially the new passive side contains almost every possible configuration.

We can discard all configurations containing symbols that only appear on one side, as using such a configuration in a solution is impossible. Luckily the active side can be pruned, cutting down the number of viable symbols even more. 

\subsection{Line superiority}

If two configurations A and B can be ordered so that $\forall i : A_i \subseteq B_i$, I call B superior to A and A inferior to B.

With this new definition we can see that the new passive side contains all configurations from the old active side wrapped in singleton sets and configurations superior to those, as

Recall that one can must be able to pick an element from each set of a new passive configuration such that

In the new passive configurations, any set can be replaced with its superset. After adding things to the sets it is still possible to choose the same elements from each set that form a configuration from the original active side.

Thus, , B can always be used instead of A; if there was some passive side that matched with A, there is a bigger one that will work with B as well. In this case, I'd say that B is \emph{superior} to A and A is \emph{inferior} to B.

Note that this is not the same as the relation $\text{configurations}(A) \subseteq \text{configurations}(B)$. "a a bcde" allows strictly less colorings than "a abc ade" but they are incomparable in terms of superiority.

\subsection{Combining lines}

My new maximization algorithm mostly consists of combining lines. Two lines can be combined by pairing up their sets and taking the union of one pair and the intersection of the rest.

The combination of two valid lines is a valid line. Proof: Let C be the combination of A and B. For each configuration c in C, Wlog. suppose the symbol that comes from the union is from A. The symbols that come from intersections are in both A and B. Thus all symbols in c are from A.

For any valid line, a superior line can be built by combining input lines. Thus the maximal lines can be built this way. The proof of this is less trivial and will be discussed later.

But even if the combining lines eventually produces the maximal lines, that doesn't mean it is an efficient way of finding them. However, it can be shown that if a line exists that isn't inferior to any of a set of lines then two of those lines can be combined to produce some line that isn't inferior to any of them. In other words, we can make progress by simply trying all combinations of two lines!

\section{Coq}

\subsection{Classical vs Intuitionistic Logic}

In Coq it is customary not to assume the law of excluded middle $\forall a : a \lor \lnot a$. This has little effect on the proof at hand, as it is about finite objects. For those a = b is equivalent to a == b and the latter is always true or false, as it can be computed.

\clearpage
\thesisbibliography{}

\bibliographystyle{plainnat}
\bibliography{correct_round_elimination}

%% Appendices
%% If you don't have appendices, remove \clearpage and \thesisappendix below.
\clearpage
\thesisappendix{}

\section{Esimerkki liitteestä\label{LiiteA}}

Kaavojen numerointi muodostaa liitteissä oman kokonaisuutensa:
\begin{align}
d \wedge A &= F, \label{liitekaava1}\\
d \wedge F &= 0. \label{liitekaava2}
\end{align}

\end{document}
